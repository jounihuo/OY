\documentclass[]{article}
\usepackage{lmodern}
\usepackage{amssymb,amsmath}
\usepackage{ifxetex,ifluatex}
\usepackage{fixltx2e} % provides \textsubscript
\ifnum 0\ifxetex 1\fi\ifluatex 1\fi=0 % if pdftex
  \usepackage[T1]{fontenc}
  \usepackage[utf8]{inputenc}
\else % if luatex or xelatex
  \ifxetex
    \usepackage{mathspec}
  \else
    \usepackage{fontspec}
  \fi
  \defaultfontfeatures{Ligatures=TeX,Scale=MatchLowercase}
\fi
% use upquote if available, for straight quotes in verbatim environments
\IfFileExists{upquote.sty}{\usepackage{upquote}}{}
% use microtype if available
\IfFileExists{microtype.sty}{%
\usepackage{microtype}
\UseMicrotypeSet[protrusion]{basicmath} % disable protrusion for tt fonts
}{}
\usepackage[margin=1in]{geometry}
\usepackage{hyperref}
\hypersetup{unicode=true,
            pdftitle={Excercise 1 R-Notebook},
            pdfborder={0 0 0},
            breaklinks=true}
\urlstyle{same}  % don't use monospace font for urls
\usepackage{color}
\usepackage{fancyvrb}
\newcommand{\VerbBar}{|}
\newcommand{\VERB}{\Verb[commandchars=\\\{\}]}
\DefineVerbatimEnvironment{Highlighting}{Verbatim}{commandchars=\\\{\}}
% Add ',fontsize=\small' for more characters per line
\usepackage{framed}
\definecolor{shadecolor}{RGB}{248,248,248}
\newenvironment{Shaded}{\begin{snugshade}}{\end{snugshade}}
\newcommand{\AlertTok}[1]{\textcolor[rgb]{0.94,0.16,0.16}{#1}}
\newcommand{\AnnotationTok}[1]{\textcolor[rgb]{0.56,0.35,0.01}{\textbf{\textit{#1}}}}
\newcommand{\AttributeTok}[1]{\textcolor[rgb]{0.77,0.63,0.00}{#1}}
\newcommand{\BaseNTok}[1]{\textcolor[rgb]{0.00,0.00,0.81}{#1}}
\newcommand{\BuiltInTok}[1]{#1}
\newcommand{\CharTok}[1]{\textcolor[rgb]{0.31,0.60,0.02}{#1}}
\newcommand{\CommentTok}[1]{\textcolor[rgb]{0.56,0.35,0.01}{\textit{#1}}}
\newcommand{\CommentVarTok}[1]{\textcolor[rgb]{0.56,0.35,0.01}{\textbf{\textit{#1}}}}
\newcommand{\ConstantTok}[1]{\textcolor[rgb]{0.00,0.00,0.00}{#1}}
\newcommand{\ControlFlowTok}[1]{\textcolor[rgb]{0.13,0.29,0.53}{\textbf{#1}}}
\newcommand{\DataTypeTok}[1]{\textcolor[rgb]{0.13,0.29,0.53}{#1}}
\newcommand{\DecValTok}[1]{\textcolor[rgb]{0.00,0.00,0.81}{#1}}
\newcommand{\DocumentationTok}[1]{\textcolor[rgb]{0.56,0.35,0.01}{\textbf{\textit{#1}}}}
\newcommand{\ErrorTok}[1]{\textcolor[rgb]{0.64,0.00,0.00}{\textbf{#1}}}
\newcommand{\ExtensionTok}[1]{#1}
\newcommand{\FloatTok}[1]{\textcolor[rgb]{0.00,0.00,0.81}{#1}}
\newcommand{\FunctionTok}[1]{\textcolor[rgb]{0.00,0.00,0.00}{#1}}
\newcommand{\ImportTok}[1]{#1}
\newcommand{\InformationTok}[1]{\textcolor[rgb]{0.56,0.35,0.01}{\textbf{\textit{#1}}}}
\newcommand{\KeywordTok}[1]{\textcolor[rgb]{0.13,0.29,0.53}{\textbf{#1}}}
\newcommand{\NormalTok}[1]{#1}
\newcommand{\OperatorTok}[1]{\textcolor[rgb]{0.81,0.36,0.00}{\textbf{#1}}}
\newcommand{\OtherTok}[1]{\textcolor[rgb]{0.56,0.35,0.01}{#1}}
\newcommand{\PreprocessorTok}[1]{\textcolor[rgb]{0.56,0.35,0.01}{\textit{#1}}}
\newcommand{\RegionMarkerTok}[1]{#1}
\newcommand{\SpecialCharTok}[1]{\textcolor[rgb]{0.00,0.00,0.00}{#1}}
\newcommand{\SpecialStringTok}[1]{\textcolor[rgb]{0.31,0.60,0.02}{#1}}
\newcommand{\StringTok}[1]{\textcolor[rgb]{0.31,0.60,0.02}{#1}}
\newcommand{\VariableTok}[1]{\textcolor[rgb]{0.00,0.00,0.00}{#1}}
\newcommand{\VerbatimStringTok}[1]{\textcolor[rgb]{0.31,0.60,0.02}{#1}}
\newcommand{\WarningTok}[1]{\textcolor[rgb]{0.56,0.35,0.01}{\textbf{\textit{#1}}}}
\usepackage{graphicx,grffile}
\makeatletter
\def\maxwidth{\ifdim\Gin@nat@width>\linewidth\linewidth\else\Gin@nat@width\fi}
\def\maxheight{\ifdim\Gin@nat@height>\textheight\textheight\else\Gin@nat@height\fi}
\makeatother
% Scale images if necessary, so that they will not overflow the page
% margins by default, and it is still possible to overwrite the defaults
% using explicit options in \includegraphics[width, height, ...]{}
\setkeys{Gin}{width=\maxwidth,height=\maxheight,keepaspectratio}
\IfFileExists{parskip.sty}{%
\usepackage{parskip}
}{% else
\setlength{\parindent}{0pt}
\setlength{\parskip}{6pt plus 2pt minus 1pt}
}
\setlength{\emergencystretch}{3em}  % prevent overfull lines
\providecommand{\tightlist}{%
  \setlength{\itemsep}{0pt}\setlength{\parskip}{0pt}}
\setcounter{secnumdepth}{0}
% Redefines (sub)paragraphs to behave more like sections
\ifx\paragraph\undefined\else
\let\oldparagraph\paragraph
\renewcommand{\paragraph}[1]{\oldparagraph{#1}\mbox{}}
\fi
\ifx\subparagraph\undefined\else
\let\oldsubparagraph\subparagraph
\renewcommand{\subparagraph}[1]{\oldsubparagraph{#1}\mbox{}}
\fi

%%% Use protect on footnotes to avoid problems with footnotes in titles
\let\rmarkdownfootnote\footnote%
\def\footnote{\protect\rmarkdownfootnote}

%%% Change title format to be more compact
\usepackage{titling}

% Create subtitle command for use in maketitle
\providecommand{\subtitle}[1]{
  \posttitle{
    \begin{center}\large#1\end{center}
    }
}

\setlength{\droptitle}{-2em}

  \title{Excercise 1 R-Notebook}
    \pretitle{\vspace{\droptitle}\centering\huge}
  \posttitle{\par}
    \author{}
    \preauthor{}\postauthor{}
    \date{}
    \predate{}\postdate{}
  

\begin{document}
\maketitle

Student: Jouni Huopana Student ID:

This is an \href{http://rmarkdown.rstudio.com}{R Markdown} Notebook for
Oulu University Data Mining Course. When you execute code within the
notebook, the results appear beneath the code.

\hypertarget{getting-your-data-into-r}{%
\subsection{1. GETTING YOUR DATA INTO
R}\label{getting-your-data-into-r}}

\begin{Shaded}
\begin{Highlighting}[]
\KeywordTok{library}\NormalTok{(psych)}
\end{Highlighting}
\end{Shaded}

Reading data in

\begin{Shaded}
\begin{Highlighting}[]
\NormalTok{mydata <-}\StringTok{ }\KeywordTok{read.csv}\NormalTok{(}\DataTypeTok{file=}\StringTok{"concretedata.csv"}\NormalTok{, }\DataTypeTok{head=}\OtherTok{TRUE}\NormalTok{, }\DataTypeTok{sep=}\StringTok{";"}\NormalTok{)}

\NormalTok{wrongdata <-}\StringTok{ }\KeywordTok{read.csv}\NormalTok{(}\DataTypeTok{file=}\StringTok{"concretedata.csv"}\NormalTok{, }\DataTypeTok{sep=}\StringTok{";"}\NormalTok{, }\DataTypeTok{dec=}\StringTok{","}\NormalTok{)}
\end{Highlighting}
\end{Shaded}

\hypertarget{simple-data-analysis}{%
\subsection{2. SIMPLE DATA ANALYSIS}\label{simple-data-analysis}}

Printing data features, type:

\begin{Shaded}
\begin{Highlighting}[]
\KeywordTok{is.data.frame}\NormalTok{(mydata)}
\end{Highlighting}
\end{Shaded}

\begin{verbatim}
## [1] TRUE
\end{verbatim}

Dimensions

\begin{Shaded}
\begin{Highlighting}[]
\KeywordTok{dim}\NormalTok{(mydata)}
\end{Highlighting}
\end{Shaded}

\begin{verbatim}
## [1] 100  12
\end{verbatim}

First 6 rows of the data

\begin{Shaded}
\begin{Highlighting}[]
\KeywordTok{head}\NormalTok{(mydata)}
\end{Highlighting}
\end{Shaded}

\begin{verbatim}
##   ID cement class    grade blast_furnace_slag fly_ash water
## 1  1  540.0     2     good                0.0       0   162
## 2  2  540.0     2 moderate                0.0       0   162
## 3  3  332.5     3     fair              142.5       0   228
## 4  4  332.5     3     good              142.5       0   228
## 5  5  198.6     1 moderate              132.4       0   192
## 6  6  266.0     1     fair              114.0       0   228
##   superplasticizer coarse_aggregate fine_aggregate age
## 1              2.5           1040.0          676.0  28
## 2              2.5           1055.0          676.0  28
## 3              0.0            932.0          594.0 270
## 4              0.0            932.0          594.0 365
## 5              0.0            978.4          825.5 360
## 6              0.0            932.0          670.0  90
##   concrete_compressive_strength
## 1                         79.99
## 2                         61.89
## 3                         40.27
## 4                         41.05
## 5                         44.30
## 6                         47.03
\end{verbatim}

More

\begin{Shaded}
\begin{Highlighting}[]
\KeywordTok{head}\NormalTok{(mydata, }\DataTypeTok{n=}\DecValTok{10}\NormalTok{)}
\end{Highlighting}
\end{Shaded}

\begin{verbatim}
##    ID cement class    grade blast_furnace_slag fly_ash water
## 1   1  540.0     2     good                0.0       0   162
## 2   2  540.0     2 moderate                0.0       0   162
## 3   3  332.5     3     fair              142.5       0   228
## 4   4  332.5     3     good              142.5       0   228
## 5   5  198.6     1 moderate              132.4       0   192
## 6   6  266.0     1     fair              114.0       0   228
## 7   7  380.0     2     good               95.0       0   228
## 8   8  380.0     2 moderate               95.0       0   228
## 9   9  266.0     3     fair              114.0       0   228
## 10 10  475.0     3     good                0.0       0   228
##    superplasticizer coarse_aggregate fine_aggregate age
## 1               2.5           1040.0          676.0  28
## 2               2.5           1055.0          676.0  28
## 3               0.0            932.0          594.0 270
## 4               0.0            932.0          594.0 365
## 5               0.0            978.4          825.5 360
## 6               0.0            932.0          670.0  90
## 7               0.0            932.0          594.0 365
## 8               0.0            932.0          594.0  28
## 9               0.0            932.0          670.0  28
## 10              0.0            932.0          594.0  28
##    concrete_compressive_strength
## 1                          79.99
## 2                          61.89
## 3                          40.27
## 4                          41.05
## 5                          44.30
## 6                          47.03
## 7                          43.70
## 8                          36.45
## 9                          45.85
## 10                         39.29
\end{verbatim}

last 6 lines

\begin{Shaded}
\begin{Highlighting}[]
\KeywordTok{tail}\NormalTok{(mydata)}
\end{Highlighting}
\end{Shaded}

\begin{verbatim}
##      ID cement class    grade blast_furnace_slag fly_ash water
## 95   95  313.3     1 moderate              262.2       0 175.5
## 96   96  425.0     1     fair              106.3       0 153.5
## 97   97  425.0     2     good              106.3       0 151.4
## 98   98  375.0     2 moderate               93.8       0 126.6
## 99   99  475.0     3     fair              118.8       0 181.1
## 100 100  469.0     3     good              117.2       0 137.8
##     superplasticizer coarse_aggregate fine_aggregate age
## 95               8.6           1046.9          611.8   7
## 96              16.5            852.1          887.1   7
## 97              18.6            936.0          803.7   7
## 98              23.4            852.1          992.6   7
## 99               8.9            852.1          781.5   7
## 100             32.2            852.1          840.5   7
##     concrete_compressive_strength
## 95                           42.8
## 96                           49.2
## 97                           46.8
## 98                           45.7
## 99                           55.6
## 100                          54.9
\end{verbatim}

Types of data

\begin{Shaded}
\begin{Highlighting}[]
\KeywordTok{str}\NormalTok{(mydata)}
\end{Highlighting}
\end{Shaded}

\begin{verbatim}
## 'data.frame':    100 obs. of  12 variables:
##  $ ID                           : int  1 2 3 4 5 6 7 8 9 10 ...
##  $ cement                       : num  540 540 332 332 199 ...
##  $ class                        : int  2 2 3 3 1 1 2 2 3 3 ...
##  $ grade                        : Factor w/ 3 levels "fair","good",..: 2 3 1 2 3 1 2 3 1 2 ...
##  $ blast_furnace_slag           : num  0 0 142 142 132 ...
##  $ fly_ash                      : int  0 0 0 0 0 0 0 0 0 0 ...
##  $ water                        : num  162 162 228 228 192 228 228 228 228 228 ...
##  $ superplasticizer             : num  2.5 2.5 0 0 0 0 0 0 0 0 ...
##  $ coarse_aggregate             : num  1040 1055 932 932 978 ...
##  $ fine_aggregate               : num  676 676 594 594 826 ...
##  $ age                          : int  28 28 270 365 360 90 365 28 28 28 ...
##  $ concrete_compressive_strength: num  80 61.9 40.3 41 44.3 ...
\end{verbatim}

Same with another set

\begin{Shaded}
\begin{Highlighting}[]
\KeywordTok{str}\NormalTok{(wrongdata)}
\end{Highlighting}
\end{Shaded}

\begin{verbatim}
## 'data.frame':    100 obs. of  12 variables:
##  $ ID                           : int  1 2 3 4 5 6 7 8 9 10 ...
##  $ cement                       : Factor w/ 31 levels "139.6","190",..: 31 31 12 12 3 5 20 20 5 28 ...
##  $ class                        : int  2 2 3 3 1 1 2 2 3 3 ...
##  $ grade                        : Factor w/ 3 levels "fair","good",..: 2 3 1 2 3 1 2 3 1 2 ...
##  $ blast_furnace_slag           : Factor w/ 25 levels "0","106.3","114",..: 1 1 7 7 6 3 24 24 3 1 ...
##  $ fly_ash                      : int  0 0 0 0 0 0 0 0 0 0 ...
##  $ water                        : Factor w/ 22 levels "126.6","137.8",..: 13 13 22 22 21 22 22 22 22 22 ...
##  $ superplasticizer             : Factor w/ 20 levels "0","10.1","10.3",..: 13 13 1 1 1 1 1 1 1 1 ...
##  $ coarse_aggregate             : Factor w/ 17 levels "1004.6","1040",..: 2 5 11 11 17 11 11 11 11 11 ...
##  $ fine_aggregate               : Factor w/ 22 levels "594","605","611.8",..: 6 6 1 1 14 5 1 1 5 1 ...
##  $ age                          : int  28 28 270 365 360 90 365 28 28 28 ...
##  $ concrete_compressive_strength: Factor w/ 93 levels "14.59","14.64",..: 90 88 39 44 60 67 58 26 63 34 ...
\end{verbatim}

\begin{Shaded}
\begin{Highlighting}[]
\KeywordTok{summary}\NormalTok{(mydata)}
\end{Highlighting}
\end{Shaded}

\begin{verbatim}
##        ID             cement          class           grade   
##  Min.   :  1.00   Min.   :139.6   Min.   :1.00   fair    :33  
##  1st Qu.: 25.75   1st Qu.:266.0   1st Qu.:1.00   good    :34  
##  Median : 50.50   Median :345.5   Median :2.00   moderate:33  
##  Mean   : 50.50   Mean   :338.7   Mean   :2.02                
##  3rd Qu.: 75.25   3rd Qu.:425.0   3rd Qu.:3.00                
##  Max.   :100.00   Max.   :540.0   Max.   :3.00                
##  blast_furnace_slag    fly_ash      water       superplasticizer
##  Min.   :  0.0      Min.   :0   Min.   :126.6   Min.   : 0.00   
##  1st Qu.: 47.5      1st Qu.:0   1st Qu.:170.1   1st Qu.: 0.00   
##  Median :114.0      Median :0   Median :228.0   Median : 0.00   
##  Mean   :117.9      Mean   :0   Mean   :199.9   Mean   : 4.67   
##  3rd Qu.:189.0      3rd Qu.:0   3rd Qu.:228.0   3rd Qu.: 9.65   
##  Max.   :282.8      Max.   :0   Max.   :228.0   Max.   :32.20   
##  coarse_aggregate fine_aggregate       age       
##  Min.   : 852.1   Min.   :594.0   Min.   :  3.0  
##  1st Qu.: 932.0   1st Qu.:594.0   1st Qu.:  3.0  
##  Median : 932.0   Median :670.0   Median : 28.0  
##  Mean   : 942.6   Mean   :708.2   Mean   :108.5  
##  3rd Qu.: 944.7   3rd Qu.:806.9   3rd Qu.:180.0  
##  Max.   :1134.3   Max.   :992.6   Max.   :365.0  
##  concrete_compressive_strength
##  Min.   : 8.06                
##  1st Qu.:34.15                
##  Median :40.81                
##  Mean   :40.18                
##  3rd Qu.:46.95                
##  Max.   :79.99
\end{verbatim}

Data names

\begin{Shaded}
\begin{Highlighting}[]
\KeywordTok{names}\NormalTok{(mydata)}
\end{Highlighting}
\end{Shaded}

\begin{verbatim}
##  [1] "ID"                            "cement"                       
##  [3] "class"                         "grade"                        
##  [5] "blast_furnace_slag"            "fly_ash"                      
##  [7] "water"                         "superplasticizer"             
##  [9] "coarse_aggregate"              "fine_aggregate"               
## [11] "age"                           "concrete_compressive_strength"
\end{verbatim}

\hypertarget{basic-statistics}{%
\subsection{3. BASIC STATISTICS}\label{basic-statistics}}

Calculating the correlations between variables in data

\begin{Shaded}
\begin{Highlighting}[]
\KeywordTok{cor}\NormalTok{(mydata[,}\DecValTok{7}\OperatorTok{:}\DecValTok{10}\NormalTok{])}
\end{Highlighting}
\end{Shaded}

\begin{verbatim}
##                         water superplasticizer coarse_aggregate
## water             1.000000000       -0.8428988      0.006406531
## superplasticizer -0.842898841        1.0000000     -0.392460650
## coarse_aggregate  0.006406531       -0.3924607      1.000000000
## fine_aggregate   -0.801410365        0.6142677     -0.139999763
##                  fine_aggregate
## water                -0.8014104
## superplasticizer      0.6142677
## coarse_aggregate     -0.1399998
## fine_aggregate        1.0000000
\end{verbatim}

Note that the correlation is meaningful only for continuous variables.
If you have missing values in your data, you may need this argument

\begin{Shaded}
\begin{Highlighting}[]
\NormalTok{na.rm=}\OtherTok{TRUE}
\end{Highlighting}
\end{Shaded}

It removes the missing values while using the selected function. You can
calculate the mean for one variable (vector) with

\begin{Shaded}
\begin{Highlighting}[]
\KeywordTok{mean}\NormalTok{(mydata}\OperatorTok{$}\NormalTok{water)}
\end{Highlighting}
\end{Shaded}

\begin{verbatim}
## [1] 199.926
\end{verbatim}

Calculating means from partly data

\begin{Shaded}
\begin{Highlighting}[]
\KeywordTok{apply}\NormalTok{(mydata[,}\OperatorTok{-}\DecValTok{4}\NormalTok{], }\DecValTok{1}\NormalTok{, median)}
\end{Highlighting}
\end{Shaded}

\begin{verbatim}
##   [1]  28.00  28.00 142.50 142.50 132.40  90.00  95.00  36.45  45.85  28.00
##  [11]  90.00  28.02  47.50  90.00  47.81  52.91  90.00  56.14  90.00  42.62
##  [21]  47.50  28.24  23.00 139.60  52.52  53.30  95.00  52.12  37.43  30.00
##  [31]  76.00 114.00 132.40  42.13 190.00 228.00  37.00  90.00  42.23 180.00
##  [41]  50.46  47.50 228.00  53.10  47.50  46.00  15.05  95.00  49.00  50.00
##  [51] 142.50 180.00  90.00  76.00  55.00  56.00  57.00  58.00  76.00  60.00
##  [61]  76.00 114.00   9.87 190.00 114.00  66.00 139.60  68.00  69.00  70.00
##  [71]  71.00  72.00  73.00  74.00  75.00  76.00  77.00  78.00  79.00  41.30
##  [81]  81.00  82.00  83.00  84.00  85.00  86.00  87.00  88.00  89.00  90.00
##  [91]  91.00  92.00  93.00  94.00  95.00  96.00  97.00  93.80  99.00 100.00
\end{verbatim}

\begin{Shaded}
\begin{Highlighting}[]
\KeywordTok{apply}\NormalTok{(mydata[,}\OperatorTok{-}\DecValTok{4}\NormalTok{], }\DecValTok{2}\NormalTok{, median)}
\end{Highlighting}
\end{Shaded}

\begin{verbatim}
##                            ID                        cement 
##                         50.50                        345.50 
##                         class            blast_furnace_slag 
##                          2.00                        114.00 
##                       fly_ash                         water 
##                          0.00                        228.00 
##              superplasticizer              coarse_aggregate 
##                          0.00                        932.00 
##                fine_aggregate                           age 
##                        670.00                         28.00 
## concrete_compressive_strength 
##                         40.81
\end{verbatim}

\begin{Shaded}
\begin{Highlighting}[]
\NormalTok{?apply}
\end{Highlighting}
\end{Shaded}

\hypertarget{selecting-data}{%
\subsection{4. SELECTING DATA}\label{selecting-data}}

Select only ten first rows

\begin{Shaded}
\begin{Highlighting}[]
\NormalTok{mydata[}\DecValTok{1}\OperatorTok{:}\DecValTok{10}\NormalTok{,}\DecValTok{2}\OperatorTok{:}\DecValTok{4}\NormalTok{]}
\end{Highlighting}
\end{Shaded}

\begin{verbatim}
##    cement class    grade
## 1   540.0     2     good
## 2   540.0     2 moderate
## 3   332.5     3     fair
## 4   332.5     3     good
## 5   198.6     1 moderate
## 6   266.0     1     fair
## 7   380.0     2     good
## 8   380.0     2 moderate
## 9   266.0     3     fair
## 10  475.0     3     good
\end{verbatim}

vector with c() if you do not want all the columns between 2 and 4

\begin{Shaded}
\begin{Highlighting}[]
\NormalTok{mydata[}\DecValTok{1}\OperatorTok{:}\DecValTok{10}\NormalTok{,}\KeywordTok{c}\NormalTok{(}\DecValTok{2}\NormalTok{,}\DecValTok{4}\NormalTok{)]}
\end{Highlighting}
\end{Shaded}

\begin{verbatim}
##    cement    grade
## 1   540.0     good
## 2   540.0 moderate
## 3   332.5     fair
## 4   332.5     good
## 5   198.6 moderate
## 6   266.0     fair
## 7   380.0     good
## 8   380.0 moderate
## 9   266.0     fair
## 10  475.0     good
\end{verbatim}

Select all observations with value \textless{}160 for variable water?

\begin{Shaded}
\begin{Highlighting}[]
\NormalTok{mydata}\OperatorTok{$}\NormalTok{water}\OperatorTok{<}\DecValTok{160}
\end{Highlighting}
\end{Shaded}

\begin{verbatim}
##   [1] FALSE FALSE FALSE FALSE FALSE FALSE FALSE FALSE FALSE FALSE FALSE
##  [12] FALSE FALSE FALSE FALSE FALSE FALSE FALSE FALSE FALSE FALSE FALSE
##  [23] FALSE FALSE FALSE FALSE FALSE FALSE FALSE FALSE FALSE FALSE FALSE
##  [34] FALSE FALSE FALSE FALSE FALSE FALSE FALSE FALSE FALSE FALSE FALSE
##  [45] FALSE FALSE FALSE FALSE FALSE FALSE FALSE FALSE FALSE FALSE FALSE
##  [56] FALSE FALSE FALSE FALSE FALSE FALSE FALSE FALSE FALSE FALSE FALSE
##  [67] FALSE FALSE FALSE  TRUE FALSE FALSE  TRUE  TRUE  TRUE FALSE  TRUE
##  [78]  TRUE  TRUE  TRUE  TRUE  TRUE  TRUE FALSE FALSE  TRUE FALSE  TRUE
##  [89] FALSE FALSE  TRUE FALSE FALSE FALSE FALSE  TRUE  TRUE  TRUE FALSE
## [100]  TRUE
\end{verbatim}

Comparing the result to this one:

\begin{Shaded}
\begin{Highlighting}[]
\NormalTok{mydata[mydata}\OperatorTok{$}\NormalTok{water}\OperatorTok{<}\DecValTok{160}\NormalTok{,]}
\end{Highlighting}
\end{Shaded}

\begin{verbatim}
##      ID cement class    grade blast_furnace_slag fly_ash water
## 70   70  485.0     3     good                0.0       0 146.0
## 73   73  425.0     2     good              106.3       0 153.5
## 74   74  425.0     2 moderate              106.3       0 151.4
## 75   75  375.0     3     fair               93.8       0 126.6
## 77   77  469.0     1 moderate              117.2       0 137.8
## 78   78  425.0     1     fair              106.3       0 153.5
## 79   79  388.6     2     good               97.1       0 157.9
## 80   80  531.3     2 moderate                0.0       0 141.8
## 81   81  425.0     3     fair              106.3       0 153.5
## 82   82  318.8     3     good              212.5       0 155.7
## 83   83  401.8     1 moderate               94.7       0 147.4
## 86   86  379.5     2 moderate              151.2       0 153.9
## 88   88  286.3     3     good              200.9       0 144.7
## 91   91  389.9     2     good              189.0       0 145.9
## 96   96  425.0     1     fair              106.3       0 153.5
## 97   97  425.0     2     good              106.3       0 151.4
## 98   98  375.0     2 moderate               93.8       0 126.6
## 100 100  469.0     3     good              117.2       0 137.8
##     superplasticizer coarse_aggregate fine_aggregate age
## 70               0.0           1120.0          800.0  28
## 73              16.5            852.1          887.1   3
## 74              18.6            936.0          803.7   3
## 75              23.4            852.1          992.6   3
## 77              32.2            852.1          840.5   3
## 78              16.5            852.1          887.1   3
## 79              12.1            852.1          925.7   3
## 80              28.2            852.1          893.7   3
## 81              16.5            852.1          887.1   3
## 82              14.3            852.1          880.4   3
## 83              11.4            946.8          852.1   3
## 86              15.9           1134.3          605.0   3
## 88              11.2           1004.6          803.7   3
## 91              22.0            944.7          755.8   3
## 96              16.5            852.1          887.1   7
## 97              18.6            936.0          803.7   7
## 98              23.4            852.1          992.6   7
## 100             32.2            852.1          840.5   7
##     concrete_compressive_strength
## 70                          71.99
## 73                          33.40
## 74                          36.30
## 75                          29.00
## 77                          40.20
## 78                          33.40
## 79                          28.10
## 80                          41.30
## 81                          33.40
## 82                          25.20
## 83                          41.10
## 86                          28.60
## 88                          24.40
## 91                          40.60
## 96                          49.20
## 97                          46.80
## 98                          45.70
## 100                         54.90
\end{verbatim}

Selecting both rows and columns

\begin{Shaded}
\begin{Highlighting}[]
\NormalTok{mydata[mydata}\OperatorTok{$}\NormalTok{water}\OperatorTok{<}\DecValTok{160}\NormalTok{,}\KeywordTok{c}\NormalTok{(}\DecValTok{1}\NormalTok{,}\DecValTok{3}\NormalTok{)]}
\end{Highlighting}
\end{Shaded}

\begin{verbatim}
##      ID class
## 70   70     3
## 73   73     2
## 74   74     2
## 75   75     3
## 77   77     1
## 78   78     1
## 79   79     2
## 80   80     2
## 81   81     3
## 82   82     3
## 83   83     1
## 86   86     2
## 88   88     3
## 91   91     2
## 96   96     1
## 97   97     2
## 98   98     2
## 100 100     3
\end{verbatim}

refering to columns by their names

\begin{Shaded}
\begin{Highlighting}[]
\NormalTok{mydata[mydata}\OperatorTok{$}\NormalTok{water}\OperatorTok{<}\DecValTok{160}\NormalTok{,}\KeywordTok{c}\NormalTok{(}\StringTok{"fly_ash"}\NormalTok{,}\StringTok{"coarse_aggregate"}\NormalTok{)]}
\end{Highlighting}
\end{Shaded}

\begin{verbatim}
##     fly_ash coarse_aggregate
## 70        0           1120.0
## 73        0            852.1
## 74        0            936.0
## 75        0            852.1
## 77        0            852.1
## 78        0            852.1
## 79        0            852.1
## 80        0            852.1
## 81        0            852.1
## 82        0            852.1
## 83        0            946.8
## 86        0           1134.3
## 88        0           1004.6
## 91        0            944.7
## 96        0            852.1
## 97        0            936.0
## 98        0            852.1
## 100       0            852.1
\end{verbatim}

Other way to do it:

\begin{Shaded}
\begin{Highlighting}[]
\NormalTok{water160 <-}\StringTok{ }\NormalTok{mydata}\OperatorTok{$}\NormalTok{water }\OperatorTok{<}\StringTok{ }\DecValTok{160}
\NormalTok{cols <-}\StringTok{ }\KeywordTok{c}\NormalTok{(}\StringTok{"fly_ash"}\NormalTok{, }\StringTok{"coarse_aggregate"}\NormalTok{)}
\NormalTok{mydata[water160, cols]}
\end{Highlighting}
\end{Shaded}

\begin{verbatim}
##     fly_ash coarse_aggregate
## 70        0           1120.0
## 73        0            852.1
## 74        0            936.0
## 75        0            852.1
## 77        0            852.1
## 78        0            852.1
## 79        0            852.1
## 80        0            852.1
## 81        0            852.1
## 82        0            852.1
## 83        0            946.8
## 86        0           1134.3
## 88        0           1004.6
## 91        0            944.7
## 96        0            852.1
## 97        0            936.0
## 98        0            852.1
## 100       0            852.1
\end{verbatim}

Subset of the original data, it can be selected with subset() function.

\begin{Shaded}
\begin{Highlighting}[]
\KeywordTok{subset}\NormalTok{(mydata, water}\OperatorTok{<}\DecValTok{160}\NormalTok{, }\KeywordTok{c}\NormalTok{(}\StringTok{"fly_ash"}\NormalTok{, }\StringTok{"coarse_aggregate"}\NormalTok{))}
\end{Highlighting}
\end{Shaded}

\begin{verbatim}
##     fly_ash coarse_aggregate
## 70        0           1120.0
## 73        0            852.1
## 74        0            936.0
## 75        0            852.1
## 77        0            852.1
## 78        0            852.1
## 79        0            852.1
## 80        0            852.1
## 81        0            852.1
## 82        0            852.1
## 83        0            946.8
## 86        0           1134.3
## 88        0           1004.6
## 91        0            944.7
## 96        0            852.1
## 97        0            936.0
## 98        0            852.1
## 100       0            852.1
\end{verbatim}

Observations that have maximum value in class

\begin{Shaded}
\begin{Highlighting}[]
\KeywordTok{subset}\NormalTok{(mydata, class}\OperatorTok{==}\KeywordTok{max}\NormalTok{(class))}
\end{Highlighting}
\end{Shaded}

\begin{verbatim}
##      ID cement class grade blast_furnace_slag fly_ash water
## 3     3  332.5     3  fair              142.5       0 228.0
## 4     4  332.5     3  good              142.5       0 228.0
## 9     9  266.0     3  fair              114.0       0 228.0
## 10   10  475.0     3  good                0.0       0 228.0
## 15   15  304.0     3  fair               76.0       0 228.0
## 16   16  380.0     3  good                0.0       0 228.0
## 21   21  427.5     3  fair               47.5       0 228.0
## 22   22  139.6     3  good              209.4       0 192.0
## 27   27  380.0     3  fair               95.0       0 228.0
## 28   28  342.0     3  good               38.0       0 228.0
## 33   33  198.6     3  fair              132.4       0 192.0
## 34   34  475.0     3  good                0.0       0 228.0
## 39   39  475.0     3  fair                0.0       0 228.0
## 40   40  237.5     3  good              237.5       0 228.0
## 45   45  427.5     3  fair               47.5       0 228.0
## 46   46  427.5     3  good               47.5       0 228.0
## 51   51  332.5     3  fair              142.5       0 228.0
## 52   52  190.0     3  good              190.0       0 228.0
## 57   57  475.0     3  fair                0.0       0 228.0
## 58   58  198.6     3  good              132.4       0 192.0
## 63   63  310.0     3  fair                0.0       0 192.0
## 64   64  190.0     3  good              190.0       0 228.0
## 69   69  190.0     3  fair              190.0       0 228.0
## 70   70  485.0     3  good                0.0       0 146.0
## 75   75  375.0     3  fair               93.8       0 126.6
## 76   76  475.0     3  good              118.8       0 181.1
## 81   81  425.0     3  fair              106.3       0 153.5
## 82   82  318.8     3  good              212.5       0 155.7
## 87   87  362.6     3  fair              189.0       0 164.9
## 88   88  286.3     3  good              200.9       0 144.7
## 93   93  337.9     3  fair              189.0       0 174.9
## 94   94  374.0     3  good              189.2       0 170.1
## 99   99  475.0     3  fair              118.8       0 181.1
## 100 100  469.0     3  good              117.2       0 137.8
##     superplasticizer coarse_aggregate fine_aggregate age
## 3                0.0            932.0          594.0 270
## 4                0.0            932.0          594.0 365
## 9                0.0            932.0          670.0  28
## 10               0.0            932.0          594.0  28
## 15               0.0            932.0          670.0  28
## 16               0.0            932.0          670.0  90
## 21               0.0            932.0          594.0 180
## 22               0.0           1047.0          806.9  28
## 27               0.0            932.0          594.0 270
## 28               0.0            932.0          670.0 180
## 33               0.0            978.4          825.5 180
## 34               0.0            932.0          594.0 270
## 39               0.0            932.0          594.0  90
## 40               0.0            932.0          594.0 180
## 45               0.0            932.0          594.0  90
## 46               0.0            932.0          594.0   7
## 51               0.0            932.0          594.0 180
## 52               0.0            932.0          670.0 180
## 57               0.0            932.0          594.0 365
## 58               0.0            978.4          825.5   3
## 63               0.0            971.0          850.6   3
## 64               0.0            932.0          670.0 270
## 69               0.0            932.0          670.0  28
## 70               0.0           1120.0          800.0  28
## 75              23.4            852.1          992.6   3
## 76               8.9            852.1          781.5   3
## 81              16.5            852.1          887.1   3
## 82              14.3            852.1          880.4   3
## 87              11.6            944.7          755.8   3
## 88              11.2           1004.6          803.7   3
## 93               9.5            944.7          755.8   3
## 94              10.1            926.1          756.7   7
## 99               8.9            852.1          781.5   7
## 100             32.2            852.1          840.5   7
##     concrete_compressive_strength
## 3                           40.27
## 4                           41.05
## 9                           45.85
## 10                          39.29
## 15                          47.81
## 16                          52.91
## 21                          41.84
## 22                          28.24
## 27                          41.15
## 28                          52.12
## 33                          41.72
## 34                          42.13
## 39                          42.23
## 40                          36.25
## 45                          41.54
## 46                          35.08
## 51                          39.78
## 52                          46.93
## 57                          41.93
## 58                           9.13
## 63                           9.87
## 64                          50.66
## 69                          40.86
## 70                          71.99
## 75                          29.00
## 76                          37.80
## 81                          33.40
## 82                          25.20
## 87                          35.30
## 88                          24.40
## 93                          24.10
## 94                          46.20
## 99                          55.60
## 100                         54.90
\end{verbatim}

Select only variables, there are two ways to do it

\begin{Shaded}
\begin{Highlighting}[]
\NormalTok{newdata1<-}\KeywordTok{subset}\NormalTok{(mydata, , }\KeywordTok{c}\NormalTok{(}\StringTok{"fly_ash"}\NormalTok{, }\StringTok{"coarse_aggregate"}\NormalTok{))}
\NormalTok{newdata2<-}\KeywordTok{subset}\NormalTok{(mydata, }\DataTypeTok{select=}\KeywordTok{c}\NormalTok{(}\StringTok{"fly_ash"}\NormalTok{, }\StringTok{"coarse_aggregate"}\NormalTok{))}
\end{Highlighting}
\end{Shaded}

\hypertarget{data-visualization}{%
\subsection{5. DATA VISUALIZATION}\label{data-visualization}}

Next, producing tables or visualizations from the data. For categorical
data, you may find the tables to be useful. For single variable

\begin{Shaded}
\begin{Highlighting}[]
\KeywordTok{table}\NormalTok{(mydata}\OperatorTok{$}\NormalTok{grade)}
\end{Highlighting}
\end{Shaded}

\begin{verbatim}
## 
##     fair     good moderate 
##       33       34       33
\end{verbatim}

or two variables

\begin{Shaded}
\begin{Highlighting}[]
\KeywordTok{table}\NormalTok{(mydata}\OperatorTok{$}\NormalTok{grade, mydata}\OperatorTok{$}\NormalTok{class)}
\end{Highlighting}
\end{Shaded}

\begin{verbatim}
##           
##             1  2  3
##   fair     16  0 17
##   good      0 17 17
##   moderate 16 17  0
\end{verbatim}

For continuous variables, plot visualization is more useful and add a
line to graph.

\begin{Shaded}
\begin{Highlighting}[]
\KeywordTok{plot}\NormalTok{(mydata}\OperatorTok{$}\NormalTok{age)}
\end{Highlighting}
\end{Shaded}

\includegraphics{Excersice_01_files/figure-latex/unnamed-chunk-28-1.pdf}

\begin{Shaded}
\begin{Highlighting}[]
\KeywordTok{plot}\NormalTok{(mydata}\OperatorTok{$}\NormalTok{cement,}\DataTypeTok{pch=}\StringTok{"."}\NormalTok{) }
\KeywordTok{lines}\NormalTok{(mydata}\OperatorTok{$}\NormalTok{cement)}
\end{Highlighting}
\end{Shaded}

\includegraphics{Excersice_01_files/figure-latex/unnamed-chunk-28-2.pdf}
Scatter plots may show interesting relationships

\begin{Shaded}
\begin{Highlighting}[]
\KeywordTok{plot}\NormalTok{(mydata}\OperatorTok{$}\NormalTok{water, mydata}\OperatorTok{$}\NormalTok{fine_aggregate)}
\end{Highlighting}
\end{Shaded}

\includegraphics{Excersice_01_files/figure-latex/unnamed-chunk-29-1.pdf}
You may add more information to the figure

\begin{Shaded}
\begin{Highlighting}[]
\KeywordTok{png}\NormalTok{(}\StringTok{"R_plot_01.png"}\NormalTok{)}
\KeywordTok{plot}\NormalTok{(mydata}\OperatorTok{$}\NormalTok{water, mydata}\OperatorTok{$}\NormalTok{fine_aggregate, }\DataTypeTok{xlab=}\StringTok{"Water content"}\NormalTok{, }\DataTypeTok{ylab=}\StringTok{"Amount of}
\StringTok{fine aggregate"}\NormalTok{, }\DataTypeTok{main=}\StringTok{"The relationship between water and fine aggregate"}\NormalTok{)}
\KeywordTok{dev.off}\NormalTok{()}
\end{Highlighting}
\end{Shaded}

\begin{verbatim}
## pdf 
##   2
\end{verbatim}

\begin{Shaded}
\begin{Highlighting}[]
\KeywordTok{plot}\NormalTok{(mydata}\OperatorTok{$}\NormalTok{water, mydata}\OperatorTok{$}\NormalTok{fine_aggregate, }\DataTypeTok{xlab=}\StringTok{"Water content"}\NormalTok{, }\DataTypeTok{ylab=}\StringTok{"Amount of}
\StringTok{fine aggregate"}\NormalTok{, }\DataTypeTok{main=}\StringTok{"The relationship between water and fine aggregate"}\NormalTok{)}
\end{Highlighting}
\end{Shaded}

\includegraphics{Excersice_01_files/figure-latex/unnamed-chunk-30-1.pdf}
Attach this figure to your exercise report. You can save it by selecting
``Export'' above the figure window, or simply Copy to Clipboard and add
to your document. There are lots of graphical parameters you can set in
the figure

\begin{Shaded}
\begin{Highlighting}[]
\NormalTok{?par}
\end{Highlighting}
\end{Shaded}

Histograms are a good way to visualize data

\begin{Shaded}
\begin{Highlighting}[]
\KeywordTok{hist}\NormalTok{(mydata}\OperatorTok{$}\NormalTok{fine_aggregate, }\DataTypeTok{breaks =} \DecValTok{15}\NormalTok{)}
\end{Highlighting}
\end{Shaded}

\includegraphics{Excersice_01_files/figure-latex/unnamed-chunk-32-1.pdf}
Another popular method is a boxplot figure

\begin{Shaded}
\begin{Highlighting}[]
\KeywordTok{boxplot}\NormalTok{(mydata}\OperatorTok{$}\NormalTok{fine_aggregate)}
\end{Highlighting}
\end{Shaded}

\includegraphics{Excersice_01_files/figure-latex/unnamed-chunk-33-1.pdf}

\begin{Shaded}
\begin{Highlighting}[]
\KeywordTok{boxplot}\NormalTok{(mydata}\OperatorTok{$}\NormalTok{fine_aggregate, mydata}\OperatorTok{$}\NormalTok{water, mydata}\OperatorTok{$}\NormalTok{cement)}
\end{Highlighting}
\end{Shaded}

\includegraphics{Excersice_01_files/figure-latex/unnamed-chunk-33-2.pdf}

\begin{Shaded}
\begin{Highlighting}[]
\KeywordTok{describe}\NormalTok{(mydata)}
\end{Highlighting}
\end{Shaded}

\begin{verbatim}
##                               vars   n   mean     sd median trimmed    mad
## ID                               1 100  50.50  29.01  50.50   50.50  37.06
## cement                           2 100 338.66 101.77 345.50  341.53 117.87
## class                            3 100   2.02   0.82   2.00    2.02   1.48
## grade*                           4 100   2.00   0.82   2.00    2.00   1.48
## blast_furnace_slag               5 100 117.88  77.16 114.00  116.79 104.89
## fly_ash                          6 100   0.00   0.00   0.00    0.00   0.00
## water                            7 100 199.93  33.01 228.00  203.83   0.00
## superplasticizer                 8 100   4.67   7.97   0.00    2.94   0.00
## coarse_aggregate                 9 100 942.58  56.60 932.00  938.74   0.00
## fine_aggregate                  10 100 708.19 107.83 670.00  697.42 112.68
## age                             11 100 108.53 128.70  28.00   89.66  37.06
## concrete_compressive_strength   12 100  40.18  11.96  40.81   40.60   9.36
##                                  min     max  range  skew kurtosis    se
## ID                              1.00  100.00  99.00  0.00    -1.24  2.90
## cement                        139.60  540.00 400.40 -0.25    -0.73 10.18
## class                           1.00    3.00   2.00 -0.04    -1.51  0.08
## grade*                          1.00    3.00   2.00  0.00    -1.52  0.08
## blast_furnace_slag              0.00  282.80 282.80  0.02    -0.98  7.72
## fly_ash                         0.00    0.00   0.00   NaN      NaN  0.00
## water                         126.60  228.00 101.40 -0.62    -1.14  3.30
## superplasticizer                0.00   32.20  32.20  1.66     2.02  0.80
## coarse_aggregate              852.10 1134.30 282.20  0.87     1.45  5.66
## fine_aggregate                594.00  992.60 398.60  0.59    -0.74 10.78
## age                             3.00  365.00 362.00  0.91    -0.68 12.87
## concrete_compressive_strength   8.06   79.99  71.93 -0.12     1.31  1.20
\end{verbatim}


\end{document}
